% !TEX root = ./article.tex

\documentclass{article}

\usepackage{mystyle}
\usepackage{myvars}



%-----------------------------

\begin{document}

	\maketitle % Insert title

	\thispagestyle{fancy} % All pages have headers and footers


%-----------------------------
%	ABSTRACT
%-----------------------------

	\begin{abstract}
		\noindent [TODO ]
	\end{abstract}

%-----------------------------
%	TEXT
%-----------------------------


	\section{Se sabe que un $1\%$ de las mujeres de $40$ años que participan en un examen rutinario tienen cáncer de mama. También se sabe que un $80\%$ de las mujeres que tienen cáncer de mama, darán positivo al hacerse una mamografía. Sin embargo, un $9.6\%$ de las mujeres que no tienen cáncer de mama darán positivo en una mamografía. En este contexto una mujer de $40$ años se somete a un examen rutinario y su mamografía da positivo. ¿Cuál es la probabilidad de que realmente tenga cáncer de mama?}
	\label{sec:e1}

		\paragraph{}
		[TODO]

	\section{Dadas dos variables aleatorias discretas, $X$ e $Y$, y dada su distribución de probabilidad conjunta que aparece en la tabla, se pide:}
	\label{sec:e2}

		\subsection{¿Cumple la distribución conjunta las propiedades de una distribución de probabilidades?}

			\paragraph{}
			[TODO]

		\subsection{¿Cuál es la probabilidad de $Pr(X = x_1)$?}

			\paragraph{}
			[TODO]

		\subsection{¿Cuáles son las distribuciones marginales de $Pr(X = x)$ y $Pr(Y = y)$?}

			\paragraph{}
			[TODO]

		\subsection{¿Verifican las distribuciones marginales las propiedades de una distribución de probabilidades?}

			\paragraph{}
			[TODO]

	\section{Utilizando el conjunto de datos \emph{weather-nominal-practica} que se proporciona, determinar la clasificación Naive Bayes de las siguientes instancias, utilizando la estimación de máxima verosimilitud (frecuencial)}
	\label{sec:e3}

		\begin{align}
			x_1 = <sunny, cool, normal, false> \\
			x_2 = <overcast, mild, high, true>
		\end{align}

		\paragraph{}
		[TODO]

	\section{Utilizando Weka y el clasificador NaiveBayes determinar la clasificación de los ejemplos anteriores, ¿Coindice con la clasificación calculada en el ejercicio anterior?}
	\label{sec:e4}

		\paragraph{}
		[TODO]

	\section{Entrenar con Weka, un clasificador Naive Bayes para el conjunto de datos \emph{weather-nominal}}
	\label{sec:e5}

		\subsection{Estimar la tasa de error cometida por el clasificador utilizando validación cruzada de 10 particiones}

			\paragraph{}
			[TODO]

		\subsection{Examinar la salida proporcionada por el Explorer y determinar cómo está estimando esta implementación de \emph{Naive Bayes} los parámetros del clasificador}

			\paragraph{}
			[TODO]

	\section{El conjunto de datos \emph{weather-nominal-X6} se ha generado repitiendo cada instancia del conjunto \emph{weather-nominal} seis veces. Entrenar con Weka un clasificador Naive Bayes para este conjunto de datos:}
	\label{sec:e6}

		\subsection{Estimar la tasa de error cometida por el clasificador utilizando validación cruzada de 10 particiones}

			\paragraph{}
			[TODO]

		\subsection{Compare esta tasa de error con la estimada en el ejercicio anterior y discuta los resultados}

			\paragraph{}
			[TODO]
%-----------------------------
%	Bibliographic references
%-----------------------------
	\nocite{garciparedes:machine-learning-bayesian-1}
	\nocite{subject:taa}
	\nocite{tool:weka}
  \bibliographystyle{alpha}
  \bibliography{bib/misc}

\end{document}
