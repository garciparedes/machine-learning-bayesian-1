% !TEX root = ./article.tex

\documentclass{article}

\usepackage{mystyle}
\usepackage{myvars}



%-----------------------------

\begin{document}

	\maketitle % Insert title

	\thispagestyle{fancy} % All pages have headers and footers


%-----------------------------
%	ABSTRACT
%-----------------------------

	\begin{abstract}
		\noindent [TODO ]
	\end{abstract}

%-----------------------------
%	TEXT
%-----------------------------


	\section{Se sabe que un $1\%$ de las mujeres de $40$ años que participan en un examen rutinario tienen cáncer de mama. También se sabe que un $80\%$ de las mujeres que tienen cáncer de mama, darán positivo al hacerse una mamografía. Sin embargo, un $9.6\%$ de las mujeres que no tienen cáncer de mama darán positivo en una mamografía. En este contexto una mujer de $40$ años se somete a un examen rutinario y su mamografía da positivo. ¿Cuál es la probabilidad de que realmente tenga cáncer de mama?}
	\label{sec:e1}

		\paragraph{}
		[TODO]

		\begin{align}
			X &= \text{Tener Cancer de mama} 										&\rightarrow \{0,1\} \\
			Y &= \text{Dar positivo al hacerse una mamografía} 	&\rightarrow \{0,1\} \\
			Z &= \text{Ser una mujer de 40 años y someterse a examen rutinario}	&\rightarrow \{0,1\}
		\end{align}

		\begin{align}
			Pr(Y = 1 | X = 1) = 0.800 \\
			Pr(Y = 1 | X = 0) = 0.096 \\
			Pr(X = 1 | Z = 1) = 0.010
 		\end{align}

		\begin{align}
			Pr(X = 1 | Z = 1, Y = 1) = Pr(X = 1) \cdot Pr(Y = 1 | X = 1) \cdot Pr(Z = 1 | X = 1)
		\end{align}

		\begin{align}
			Pr(Y = 1 | X = 1) = 0.800 \\
			Pr(Y = 1 | X = 0) = 0.096 \\
			Pr(X = 1) = 0.010
		\end{align}


		\begin{align}
			Pr(X = 1 | Y = 1) = Pr(X = 1) \cdot Pr(Y = 1 | X = 1) = 0.010 \cdot 0.800 = 0.008
		\end{align}

	\section{Dadas dos variables aleatorias discretas, $X$ e $Y$, y dada su distribución de probabilidad conjunta que aparece en la tabla, se pide:}
	\label{sec:e2}

	\begin{table}
		\centering
		\begin{tabular}{ | c || c | c | c | c | c |}
			\hline
			 					& $x_1$ 	& $x_2$ 	& $x_3$ 	& $x_4$ 	& $Pr(Y)$ \\ \hline \hline
				$y_1$ 	&	$2/16$	&	$1/16$	&	$1/16$	&	$1/16$	&	$5/16$ 	\\ \hline
				$y_2$ 	&	$1/16$	&	$2/16$	&	$2/16$	&	$1/16$	&	$6/16$ 	\\ \hline
				$y_3$ 	&	$1/16$	&	$1/16$	&	$1/16$	&	$0$			&	$3/16$ 	\\ \hline
				$y_4$ 	&	$0$			&	$2/16$	&	$0$			&	$0$			&	$2/16$ 	\\ \hline
				$Pr(X)$ &	$4/16$	&	$6/16$	&	$4/16$	&	$2/16$	&	$16/16$ \\
			 \hline
		\end{tabular}
		\caption{Frecuencias relativas de la distribución de probabilidad conjunta de $X$ e $Y$}
		\label{table:e2}

	\end{table}

		\subsection{¿Cumple la distribución conjunta las propiedades de una distribución de probabilidades?}

			\paragraph{}
			[TODO]

		\subsection{¿Cuál es la probabilidad de $Pr(X = x_1)$?}

			\paragraph{}
			[TODO]

		\subsection{¿Cuáles son las distribuciones marginales de $Pr(X = x)$ y $Pr(Y = y)$?}

			\paragraph{}
			[TODO]

		\subsection{¿Verifican las distribuciones marginales las propiedades de una distribución de probabilidades?}

			\paragraph{}
			[TODO]

	\section{Utilizando el conjunto de datos \emph{weather-nominal-practica} que se proporciona, determinar la clasificación Naive Bayes de las siguientes instancias, utilizando la estimación de máxima verosimilitud (frecuencial)}
	\label{sec:e3}

		\begin{align}
			x_1 = <sunny, cool, normal, false> \\
			x_2 = <overcast, mild, high, true>
		\end{align}

		\begin{align}
			Pr(play = yes) &= 9/14 \\
			Pr(play = no) &= 5/14
		\end{align}
		\begin{align}
			Pr(outlook = sunny 		| play = yes) &= 2/9 \\
			Pr(outlook = overcast | play = yes) &= 2/9 \\
			Pr(outlook = rainy 		| play = yes) &= 5/9 \\
			Pr(outlook = sunny 		| play = no) &= 3/5 \\
			Pr(outlook = overcast | play = no) &= 0/5 \\
			Pr(outlook = rainy 		| play = no) &= 2/5
		\end{align}

		\begin{align}
			Pr(temperature = hot 	| play = yes) &= 1/9 \\
			Pr(temperature = mild | play = yes) &= 4/9 \\
			Pr(temperature = cool | play = yes) &= 4/9 \\
			Pr(temperature = hot 	| play = no) &= 2/5 \\
			Pr(temperature = mild | play = no) &= 2/5 \\
			Pr(temperature = cool | play = no) &= 1/5
		\end{align}

		\begin{align}
			Pr(humidity = high 		| play = yes) &= 3/9 \\
			Pr(humidity = normal 	| play = yes) &= 6/9 \\
			Pr(humidity = high		| play = no) &= 4/5 \\
			Pr(humidity = normal 	| play = no) &= 1/5
		\end{align}

		\begin{align}
			Pr(windy = true 	| play = yes) &= 4/9 \\
			Pr(windy = false 	| play = yes) &= 5/9 \\
			Pr(windy = true 	| play = no) &= 3/5 \\
			Pr(windy = false 	| play = no) &= 2/5
		\end{align}


		\begin{align}
			Pr(outlook = sunny, temperature = cool, humidity = normal, windy = false | play = yes) = \\
			Pr(play = yes) \cdot Pr(outlook = sunny | play = yes) \cdot Pr(temperature = hot 	| play = yes) \cdot \\
			Pr(humidity = normal 	| play = yes) \cdot Pr(windy = false 	| play = yes) = \\
			9/14 \cdot 2/9 \cdot 4/9 \cdot 6/9 \cdot 5/9 = 40/1701 = 0.02351557
		\end{align}

		\begin{align}
			Pr(outlook = sunny, temperature = cool, humidity = normal, windy = false | play = no) = \\
			Pr(play = no) \cdot Pr(outlook = sunny | play = no) \cdot Pr(temperature = hot 	| play = no) \cdot \\
			Pr(humidity = normal 	| play = no) \cdot Pr(windy = false | play = no) =\\
			 5/14 \cdot 3/5 \cdot 1/5 \cdot 1/5 \cdot  2/5 = 3/875 = 0.003428571
		\end{align}


		\begin{align}
			Pr(outlook = overcast, temperature = mild, humidity = high, windy = true | play = yes) = \\
			Pr(play = yes) \cdot Pr(outlook = overcast | play = yes) \cdot Pr(temperature = mild | play = yes) \cdot \\
			Pr(humidity = high 	| play = yes) \cdot Pr(windy = true 	| play = yes) = \\
			9/14 \cdot 2/9 \cdot 4/9 \cdot 3/9 \cdot 4/9 = 16/1701 = 0.00940623
		\end{align}

		\begin{align}
			Pr(outlook = overcast, temperature = cool, humidity = high, windy = true | play = no) = \\
			Pr(play = no) \cdot Pr(outlook = overcast | play = no) \cdot Pr(temperature = mild | play = no) \cdot \\
			Pr(humidity = high 	| play = no) \cdot Pr(windy = true | play = no) =\\
			5/14 \cdot 0/5 \cdot 2/5 \cdot 4/5 \cdot  4/5 = 0
		\end{align}

		\paragraph{}
		[TODO]

	\section{Utilizando Weka y el clasificador NaiveBayes determinar la clasificación de los ejemplos anteriores, ¿Coindice con la clasificación calculada en el ejercicio anterior?}
	\label{sec:e4}

		\paragraph{}
		[TODO]

	\section{Entrenar con Weka, un clasificador Naive Bayes para el conjunto de datos \emph{weather-nominal}}
	\label{sec:e5}

		\subsection{Estimar la tasa de error cometida por el clasificador utilizando validación cruzada de 10 particiones}

			\paragraph{}
			[TODO]

		\subsection{Examinar la salida proporcionada por el Explorer y determinar cómo está estimando esta implementación de \emph{Naive Bayes} los parámetros del clasificador}

			\paragraph{}
			[TODO]

	\section{El conjunto de datos \emph{weather-nominal-X6} se ha generado repitiendo cada instancia del conjunto \emph{weather-nominal} seis veces. Entrenar con Weka un clasificador Naive Bayes para este conjunto de datos:}
	\label{sec:e6}

		\subsection{Estimar la tasa de error cometida por el clasificador utilizando validación cruzada de 10 particiones}

			\paragraph{}
			[TODO]

		\subsection{Compare esta tasa de error con la estimada en el ejercicio anterior y discuta los resultados}

			\paragraph{}
			[TODO]
%-----------------------------
%	Bibliographic references
%-----------------------------
	\nocite{garciparedes:machine-learning-bayesian-1}
	\nocite{subject:taa}
	\nocite{tool:weka}
  \bibliographystyle{alpha}
  \bibliography{bib/misc}

\end{document}
